\vspace*{1.2cm}
\section*{Prefazione}

Questo testo è una rielaborazione di appunti presi durante un corso di Matematica Numerica. Si rivolge a tutti quegli studenti che si stanno affacciando per la prima volta ai metodi numerici e vogliono avere delle basi solide e dei concetti chiari, col fine di affrontare al meglio le applicazioni future. 

Essendo una raccolta di appunti, non si pone come un sostituto di libri di testo maggiori e più approfonditi, né delle lezioni frontali con il docente, il quale rimane una risorsa fondamentale per scavare a fondo della materia. Questo libro si pone invece come un supporto ulteriore che affianca gli strumenti sopracitati e accompagna lo studente nei primi passi all'interno della Matematica Numerica.

Ringraziamo l'autore del libro, new entry nel nostro team, per aver realizzato un lavoro completo e approfondito, e per la fiducia che ci ha accordato nella pianificazione della sua revisione e pubblicazione.
Essenziali sono stati anche i contributi di Bruno Guindani, nella forma di revisione contenutistica e stilistica; di Aron Wussler, che si è occupato della revisione grafica; Fabrizio Bernardi per un'altra revisione contenutistica più approfondita; e Gabriele Gabrielli per la copertina dell'opera.

Siamo orgogliosi che il progetto e l'idea del libro ``Appunti di Probabilità'', da noi realizzato ormai quattro anni fa, possa continuare in altre materie, a beneficio di tanti nuovi studenti, e che possa espandere la nostra piccola comunità di nerd di \LaTeX. \\
\begin{flushright}
\textit{L'associazione Fubini-Tonelli}\hspace*{0.5cm}
\end{flushright}