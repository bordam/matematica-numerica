%!TEX root = ../main.tex
\quad \chapter{Risoluzione di sistemi lineari con metodi diretti}
% \textit{[Lezione 3 (11-03-2020)]}

Ci soffermeremo dapprima su metodi numerici per la risoluzione di sistemi lineari. Supponiamo siano assegnate la matrice $\displaystyle A\in \mathbb{R}^{n\times n}$ e il vettore $ \mathbf{b} \in \mathbb{R}^{n}$. Si vuole determinare il vettore $\displaystyle \x \in \mathbb{R}^{n}$ tale che
\begin{equation*}
A\mathbf{x=b}.
\end{equation*}
Possiamo scrivere il sistema in forma estesa come
\begin{equation*}
\sum ^{n}_{j=1} a_{ij} x_{j} =b_{i} ,\quad \forall i=1,\dotsc ,n,\quad a_{ij} =(A)_{ij} .
\end{equation*}
Siamo interessati a problemi \textit{ben posti}, ossia in cui la soluzione esiste ed è unica. Le condizioni affinché ciò sia verificato possono essere diverse, tutte equivalenti tra loro.
\begin{theorem}
    La soluzione $\displaystyle \x \in \mathbb{R}^{n}$ del sistema lineare $\displaystyle A\x=\mathbf{b}$ esiste ed è unica se e solo se una delle seguenti equivalenti condizioni è soddisfatta:
    \begin{itemize}
        \item $\det(A) \neq 0$, ovvero la matrice è invertibile;
        \item $\operatorname{rank}(A) =n$, ovvero il suo rango è massimo;
        \item $A\x =\mathbf{0} \Leftrightarrow \x =\mathbf{0}$, ovvero l'unica soluzione del cosiddetto sistema omogeneo è $\mathbf{0}$.
    \end{itemize}
    \label{thm:esistenza-unicita-sl}
\end{theorem}

Nel corso dei vari capitoli, molti risultati coinvolgeranno le matrici simmetriche e definite positive (SDP).
Ricordiamo il concetto di definita positività, di cui il lettore può trovare un approfondimento nell'appendice \ref{thm:matrice-definita-positiva-negativa}.
\begin{definition}
  [Matrice definita positiva/negativa]
  \index{matrice!definita positiva}
  \index{matrice!definita negativa}
  Una matrice quadrata $A\in \mathbb{R} ^{n\times n} $ si dice definita positiva (risp. negativa) se $\x^{T} A\x > 0$ (risp. $< 0$) per ogni $\x \neq \mathbf{0}$, $\x \in \mathbb{R}^{n}$.
\end{definition}
Per questo tipo di matrici, riportiamo un primo importante risultato.
\begin{theorem}
    Se $A\in \mathbb{R} ^{n\times n} $ è una matrice quadrata definita positiva, allora è invertibile.
    \label{thm:A-DP-invertibile}
\end{theorem}
Ciò si può giustificare pensando che il prodotto degli autovalori della matrice coincide col suo determinante, e se essa è definita positiva, tutti gli autovalori sono strettamente positivi, quindi il determinante è non nullo.

Per il resto della trattazione considereremo solo matrici per cui $\displaystyle A\mathbf{x=b}$ ammette una e una sola soluzione, per avere a che fare solo con problemi ben posti.

\begin{definition}
  [Matrice triangolare]
  \index{matrice!triangolare}
  Una matrice quadrata $A\in \mathbb{R}^{n\times n}$ si dice triangolare superiore (risp. inferiore) se $a_{ij} =0$ per ogni $i >j$ (risp. $i< j$), cioè se gli elementi sotto (risp. sopra) la diagonale si annullano.
\end{definition}

\section{Regola di Cramer}

È una regola che permette di calcolare ogni componente della soluzione di un certo sistema lineare $A\x=\mathbf{b}$
\begin{equation*}
x_{i} =\frac{\det( A_{i})}{\det(A)} \ \ \forall i=1,\dotsc ,n
\end{equation*}
dove
\begin{equation*}
A_{i} =[ a_{1} ,a_{2} ,\dotsc ,a_{i-1}, b,a_{i+1}  ,\dotsc ,a_{n}],
\end{equation*}
ovvero la matrice ottenuta sostituendo la $i$-esima colonna della matrice $A$ con il termine noto.

Questo algoritmo è inutilizzabile nella pratica per il suo eccessivo carico computazionale:
il costo di calcolare il determinante di una matrice $n\times n$ è $\approx n!$ operazioni, e questa regola richiede il calcolo di $n+1$ determinanti, ossia $n+1$ operazioni di complessità fattoriale.

Cogliamo l'occasione per far notare un aspetto particolarmente importante nella risoluzione di sistemi lineari. Il lettore forse si sarà chiesto perché, nel voler risolvere il sistema $A\x=\mathbf{b}$, non è stato menzionato il banale passaggio di inversione della matrice che permette di scrivere la soluzione come $\x=A^{-1}\mathbf{b}$. Il calcolo della matrice inversa è un'operazione che richiede la computazione di numerosi determinanti.
Tale operazione operazione è estremamente costosa, come appena mostrato.
Pertanto, diventa chiara la ragione per cui, salvo specifiche esigenze, si dovrebbe evitare il calcolo di una matrice inversa.
Per compiti come la risoluzione dei sistemi lineari, sono preferibili i metodi presentati in seguito, i quali consentono di determinare la soluzione, o una sua ragionevole approssimazione, senza passare per la matrice inversa.

\section{Metodi diretti: Fattorizzazione LU}
\label{sec:fattorizzazione-lu}

Supponiamo di saper decomporre la matrice $\displaystyle A$ in questo modo:
\begin{equation}
A=LU \quad L, U \in \mathbb{R}^{n\times n}.
\label{eq:fattorizzazione-LU}
\end{equation}
dove $\displaystyle L$ e $\displaystyle U$ sono matrici triangolari, rispettivamente triangolare inferiore (Lower) e superiore (Upper).
Risolvere $\displaystyle A\x =\mathbf{b}$ è equivalente a risolvere
\begin{equation*}
L\underbrace{U\x}_{\y} =\mathbf{b} \ \ \Leftrightarrow \ \ \begin{cases}
L\y =\mathbf{b}\\
U\x =\y.
\end{cases}
\end{equation*}
Se avessimo accesso a un algoritmo poco costoso per risolvere questi due sistemi triangolari, abbatteremmo il costo computazionale totale rispetto alla regola di Cramer.

\subsection{Algoritmo di sostituzione in avanti}
\index{algoritmo!sostituzione in avanti}
Supponiamo che $\displaystyle L$ sia triangolare inferiore:
\begin{equation}
\underbrace{\begin{bmatrix}
l_{11} & 0 & \dotsc  & 0\\
l_{21} & l_{22} & \ddots  & \vdots \\
\vdots  & \vdots  & \ddots  & 0\\
l_{n1} & l_{n2} & \dotsc  & l_{nn}
\end{bmatrix}}_{L}\underbrace{\begin{bmatrix}
y_{1}\\
y_{2}\\
\vdots \\
y_{n}
\end{bmatrix}}_{\y} =\underbrace{\begin{bmatrix}
b_{1}\\
b_{2}\\
\vdots \\
b_{n}
\end{bmatrix}}_{\mathbf{b}}.
\label{eq:Ly-b}
\end{equation}
\textbf{Osservazione. }$\displaystyle l_{11} \neq 0,l_{22} \neq 0,\dotsc ,l_{nn} \neq 0$ altrimenti $\displaystyle L$ sarebbe singolare, quindi anche $\displaystyle A$ sarebbe singolare\footnote{In una matrice triangolare, $L$ in questo caso, gli autovalori sono gli elementi sulla diagonale, e in generale il determinante è il prodotto degli autovalori. Se un elemento della diagonale di $L$ fosse nullo, allora anche il determinante lo sarebbe. Essendo $A=LU$, per il teorema di Binet $\operatorname{det}A=\operatorname{det}L\cdot\operatorname{det}U$, quindi sarebbe singolare anche $A$.}
e non rispetterebbe la condizione per l'esistenza e unicità della soluzione. Vediamo ora cosa succede alle varie equazioni risolutive del sistema \eqref{eq:Ly-b}
\begin{itemize}
\item alla $1^{a}$ equazione:
\begin{equation*}
l_{11} y_{1} =b_{1} \ \Rightarrow \ y_{1} =\frac{b_{1}}{l_{11}};
\end{equation*}
\item alla $2^{a}$ equazione, sfruttando il calcolo appena svolto per $y_{1}$:
\begin{equation*}
l_{21} y_{1} +l_{22} y_{2} =b_{2} \ \Rightarrow \ y_{2} =\frac{1}{l_{22}}( b_{2} -l_{21} y_{1});
\end{equation*}
\item alla $n$-esima equazione, iterando il procedimento:
\begin{equation*}
l_{n1} y_{1} +l_{n2} y_{2} +\cdots +l_{nn} y_{n} =b_{n} \ \Rightarrow \ y_{n} =\frac{1}{l_{nn}}\left[ b_{n} -\sum ^{n-1}_{j=1} l_{nj} y_{j}\right].
\end{equation*}
\end{itemize}

Riscriviamo ora questo algoritmo sotto forma di pseudocodice:

\begin{algo}
    $y_{1} = \frac{b_{1} }{l_{11} }  $\;
    \For{$i=2$ \KwTo $n$}{
        $y_{i}=\frac{1}{l_{ii}}\left( b_{i} -\sum\limits ^{i-1}_{j=1} l_{ij} y_{j}\right)$\;
        \If{criterio di arresto}{
            termina algoritmo\;
        }
    }
    \caption{Algoritmo di sostituzione in avanti}
\end{algo}

\textit{Osservazione.} Determiniamo il costo computazionale dell'algoritmo:
\begin{itemize}
\item $1$ divisione fuori dal ciclo;
\item per ogni passo del ciclo $( i-1)$ prodotti, $( i-1)$ somme e $1$ divisione per il calcolo di $y_{i}$.
\end{itemize}

Quindi il costo totale è:
\begin{align*}
  1+\sum ^{n}_{i=2}[( i-1) +( i-1) +1] &= 1+\sum ^{n}_{i=2}[ 2i-1]\\
 &= 1+\sum ^{n}_{i=1}[ 2i-1] -1 \\
 &= \sum ^{n}_{i=1}[ 2i-1]\\
 &= 2\sum ^{n}_{i=1} i-n\\
 &= 2\frac{n( n+1)}{2} -n \approx n^{2} \ \text{operazioni},
\end{align*}
che risulta molto inferiore al $n!$ della regola di Cramer.
In generale, $n^2$ è la soglia minima possibile per il numero di operazioni di un algoritmo di algebra matriciale, visto che è lo stesso ordine di grandezza del numero di elementi della matrice, ed è pertanto ottimale per un metodo numerico.

\subsection{Algoritmo di sostituzione all'indietro}
\index{algoritmo!sostituzione all'indietro}
Supponiamo invece che la matrice $\displaystyle U$ sia triangolare superiore:
\begin{equation}
\underbrace{\begin{bmatrix}
u_{11} & u_{12} & \dotsc  & u_{1n}\\
0 & u_{22} & \dotsc  & u_{2n}\\
\vdots  & \ddots  & \ddots  & \vdots \\
0 & \dotsc  & 0 & u_{nn}
\end{bmatrix}}_{U}\underbrace{\begin{bmatrix}
x_{1}\\
x_{2}\\
\vdots \\
x_{n}
\end{bmatrix}}_{\x} =\underbrace{\begin{bmatrix}
y_{1}\\
y_{2}\\
\vdots \\
y_{n}
\end{bmatrix}}_{\y}.
\label{eq:Ux-y}
\end{equation}
\textit{Osservazione.} $\displaystyle u_{11} \neq 0,u_{22} \neq 0,\dotsc ,u_{nn} \neq 0$, per le stesse ragioni enunciate per $L$.

L'idea è fare la stessa sostituzione della sezione precedente, ma andando a ritroso partendo dall'ultima riga:

\begin{algo}
    $x_{n} =\frac{y_{n}}{u_{nn}}$\;
    \For{$i=n-1$ \KwTo $1$}{
        $x_{i} =\frac{1}{u_{ii}}\left( y_{i} -\sum\limits ^{n}_{j=i+1} u_{ij} x_{j}\right)$\;
        \If{criterio di arresto}{
            termina algoritmo\;
        }
    }
    \caption{Algoritmo di sostituzione all'indietro}
\end{algo}
\textit{Osservazione.} Determiniamo il costo computazionale dell'algoritmo.
Ci aspettiamo che sia lo stesso della sostituzione in avanti:
\begin{itemize}
\item $1$ divisione fuori dal ciclo;
\item per ogni passo del ciclo $( n-i)$ prodotti, $( n-i)$ somme e $1$ divisione per il calcolo di $x_{i}$.
\end{itemize}

Il costo è dunque:
\begin{align*}
  1+\sum ^{1}_{i=n-1}[( n-i) +( n-i) +1] &= 1+\sum ^{n-1}_{i=1}[ 2n-2i+1]\\
  &=1+2n( n-1) -2\sum ^{n-1}_{i=1} i+( n-1)\\
  &=1+2n( n-1) -2\frac{( n-1) n}{2} +( n-1)\\
  &=1+2n^{2} -2n-n^{2} +n+n-1=n^{2} \ \text{operazioni}
\end{align*}

Ricapitolando:
\begin{equation*}
A\x =\mathbf{b} \ \Leftrightarrow \left\{\begin{array}{ c }
L\y =\mathbf{b}\\
U\x =\y.
\end{array}\right.
\end{equation*}
Se riuscissimo a trovare la scomposizione $\displaystyle A=LU$ dove $\displaystyle L$ è una matrice triangolare inferiore (che si risolve con sostituzione in avanti) e $\displaystyle U$ triangolare superiore (che si risolve con sostituzione all'indietro), riusciremmo a risolvere il sistema con solo approssimativamente $\displaystyle \approx 2n^{2}$ operazioni, aggirando l'enorme costo della regola di Cramer.

\subsection{Come trovare la fattorizzazione LU}

%Siamo capaci di fattorizzare $\displaystyle A$ in questo modo?

\textit{Esempio.}
\begin{equation*}
\begin{bmatrix}
a_{11} & a_{12}\\
a_{21} & a_{22}
\end{bmatrix} =\begin{bmatrix}
l_{11} & 0\\
l_{21} & l_{22}
\end{bmatrix}\begin{bmatrix}
u_{11} & u_{12}\\
0 & u_{22}
\end{bmatrix}
\end{equation*}
in questo sistema abbiamo $\displaystyle 3+3$ incognite nel membro destro e i $\displaystyle 4$ vincoli del membro sinistro. Questo problema è irrisolvibile in quanto sottodimensionato: abbiamo troppe incognite rispetto ai vincoli, il che produrrebbe infinite fattorizzazioni possibili.
Dobbiamo dunque eliminare almeno $2$ incognite.
Per esempio, possiamo fissare gli elementi $\displaystyle l_{11} =1,l_{22} =1$.
Avendo ora incognite e vincoli in egual numero, siamo nelle condizioni di poter trovare una soluzione, e non ci resta che svolgere i calcoli per trovarla.
\begin{definition}
[Fattorizzazione $\displaystyle LU$]
\index{fattorizzazione!LU}
Una matrice quadrata $\displaystyle A\in \mathbb{R}^{n\times n}$ ammette una fattorizzazione $\displaystyle LU$ se esistono $\displaystyle L\in \mathbb{R}^{n\times n}$ matrice triangolare inferiore tale che $\displaystyle l_{ii} =1\ \forall i=1,\dotsc ,n$ e $\displaystyle U\in \mathbb{R}^{n\times n}$ matrice triangolare superiore tali che $\displaystyle A=LU$.
\end{definition}
I valori degli elementi sulla diagonale principale di $\displaystyle L$ sono irrilevanti a patto che non siano nulli.
Per ragioni di stabilità numerica degli algoritmi, li scegliamo uguali a $\displaystyle 1$.
Ricordiamo infatti che gli autovalori di una matrice triangolare sono i valori sulla diagonale principale.
\begin{definition}
[Sottomatrici principali]
\index{sottomatrici principali}
Data una matrice quadrata $A \in \mathbb{R}^{n\times n}$, si definiscono \textbf{sottomatrici principali} di ordine $k$, con $k=1,\dots,n$, le sottomatrici quadrate di dimensione $k$ ottenute eliminando righe e colonne \emph{dello stesso indice}. Più in particolare, si definisce \textbf{sottomatrice principale di testa}\index{sottomatrice principale di testa} di ordine $k$ quella sottomatrice principale ottenuto eliminando le ultime $(n-k)$ righe e colonne, con $1\le k\le n$.
\end{definition}
\begin{definition}
    [Minori principali]
    \index{minori principali}
    Data una matrice quadrata $A \in \mathbb{R}^{n\times n}$, si definisce \textbf{minore principale}\index{minore principale} di ordine $k$ il determinante della sottomatrice principale di ordine $k$. Più in particolare, si definisce \textbf{minore principale di testa}\index{minore principale di testa} (o \textbf{minore di Nord-Ovest})\index{minore di nord-ovest} il determinante della sottomatrice principale di testa di ordine $k$.
\end{definition}
Naturalmente quindi una matrice quadrata di dimensione $n$ ha esattamente $n$ minori principali di testa.
\begin{theorem}
[Esistenza e unicità della fattorizzazione $\displaystyle LU$]
\index{fattorizzazione!LU!esistenza e unicità}
Sia $\displaystyle A\in \mathbb{R}^{n\times n}$ una matrice. La matrice $\displaystyle A$ ammette una e una sola fattorizzazione $\displaystyle LU$ se e solo se tutti i minori principali di testa di ordine $\displaystyle i=1,\dotsc ,n-1$ sono non nulli.
\label{thm:esist-unicita-LU}
\end{theorem}
\textit{Esempio.}
\begin{equation*}
A=\begin{bmatrix}
1 & 2 & 3\\
4 & 5 & 6\\
0 & 7 & 1
\end{bmatrix} =\begin{bmatrix}
1 & 0 & 0\\
l_{21} & 1 & 0\\
l_{31} & l_{32} & 1
\end{bmatrix}\begin{bmatrix}
u_{11} & u_{12} & u_{13}\\
0 & u_{22} & u_{23}\\
0 & 0 & u_{33}
\end{bmatrix}
\end{equation*}
Le sue sottomatrici principali di testa sono:
\begin{equation*}
A_{1} =[ 1] \qquad A_{2} =\begin{bmatrix}
1 & 2\\
4 & 5
\end{bmatrix} \qquad A=\begin{bmatrix}
1 & 2 & 3\\
4 & 5 & 6\\
0 & 7 & 1
\end{bmatrix}.
\end{equation*}
Essendo tutti non singolari, la matrice soddisfa le ipotesi del teorema \ref{thm:esist-unicita-LU}.
Per determinare i coefficienti delle matrici $L$ e $U$ si scrivono tutti i prodotti riga per colonna e si risolvono i sistemi:
\begin{align*}
u_{11} & =1\\
u_{12} & =2\\
u_{13} & =3\\
l_{21} u_{11} & =4\ \ \Rightarrow \ \ l_{21} =4\\
l_{21} u_{12} +u_{22} & =5\ \ \Rightarrow \ \ u_{22} =5-4\cdot 2=-3\\
l_{21} u_{13} +u_{23} & =6\ \ \Rightarrow \ \ u_{23} =6-4\cdot 3=-6\\
l_{31} u_{11} & =0\ \ \Rightarrow \ \ l_{31} =0\\
l_{31} u_{12} +l_{32} u_{22} & =7\ \ \Rightarrow \ \ l_{32} =-7/3\\
l_{31} u_{13} +l_{32} u_{23} +u_{33} & =1\ \ \Rightarrow \ \ u_{33} =1-( -7/3)( -6) =-13.
\end{align*}
Pertanto:
\begin{equation*}
A=\begin{bmatrix}
1 & 2 & 3\\
4 & 5 & 6\\
0 & 7 & 1
\end{bmatrix} =\begin{bmatrix}
1 & 0 & 0\\
4 & 1 & 0\\
0 & -\frac{7}{3} & 1
\end{bmatrix}\begin{bmatrix}
1 & 2 & 3\\
0 & -3 & -6\\
0 & 0 & -13
\end{bmatrix}.
\end{equation*}
Richiamiamo ora una definizione dall'algebra lineare che ci sarà utile nelle prossime pagine.
\begin{definition}
  [Predominanza diagonale stretta]
  \index{matrice!predominanza diagonale stretta}
  Una matrice $A\in \mathbb{R} ^{n\times n} $ si dice:
  \begin{itemize}
      \item a \textbf{predominanza diagonale stretta per righe} se
      \begin{equation*}
          |a_{ii} | >\sum ^{n}_{\substack{j=1\\j\neq i}} |a_{ij} |,\quad \forall i=1,\dotsc ,n
      \end{equation*}
      cioè se ogni elemento della diagonale è in modulo maggiore strettamente della somma degli altri elementi sulla sua stessa riga.
      \item a \textbf{predominanza diagonale stretta per colonne} se
      \begin{equation*}
          |a_{jj} | >\sum ^{n}_{\substack{i=1\\i\neq j}} |a_{ij} |,\quad \forall j=1,\dotsc ,n.
      \end{equation*}
  \end{itemize}
\end{definition}

\textit{Esempio.}
La matrice
\begin{equation*}
    \begin{bmatrix}
    -3 & 1 & 1\\
    0 & 2 & 1\\
    1 & 0 & -4
    \end{bmatrix}
\end{equation*}
è a predominanza diagonale stretta per righe. Infatti:
\begin{equation*}
    \begin{aligned}
    |-3| &  >|1|+|1|\\
    |+2| &  >|0|+|1|\\
    |-4| &  >|1|+|0|,
    \end{aligned}
\end{equation*}
mentre la matrice
\begin{equation*}
    \begin{bmatrix}
    5 & -1 & -2\\
    -1 & 2 & -1\\
    2 & -1 & 4
    \end{bmatrix}
\end{equation*}
non lo è. Infatti:
\begin{equation*}
    \begin{aligned}
    | 5|  &  >| -1| +|-2| \\
    | 2|  & \ngtr | -1| +| -1| \\
    | 4|  &  >| +2| +| -1|.
    \end{aligned}
\end{equation*}
\textit{Esempio.}
La matrice
\begin{equation*}
    \begin{bmatrix}
    3 & 1 & 1\\
    1 & 2 & 1\\
    0 & 0 & 3
    \end{bmatrix}
\end{equation*}
è a predominanza diagonale stretta per colonne. Infatti:
\begin{equation*}
    \begin{aligned}
    |3| &  >|1|+|0|\\
    |2| &  >|1|+|0|\\
    |3| &  >|1|+|1|.
    \end{aligned}
\end{equation*}

%\textit{Esempio.}
%\begin{equation*}
%\begin{array}{ l l l }
%\begin{bmatrix}
%    -3 & 1 & 1\\
%    0 & 2 & 1\\
%    1 & 0 & -4
%    \end{bmatrix} & \begin{aligned}
%|-3| &  >|1|+|1|\\
%|+2| &  >|0|+|1|\\
%|-4| &  >|1|+|0|
%\end{aligned} & \text{è a p.d.s.r}\\
%\begin{bmatrix}
%2 & -1 & 0\\
%-1 & 2 & -1\\
%0 & -1 & 2
%\end{bmatrix} & \begin{aligned}
%| 2|  &  >| -1| +| 0| \\
%| 2|  & \ngtr | -1| +| -1| \\
%| 2|  &  >| 0| +| -1|
%\end{aligned} & \text{non è a p.d.s.r}
%\end{array}
%\end{equation*}
%
%\textit{Esempio.}
%\begin{equation*}
%\begin{array}{ l l l }
%\begin{bmatrix}
%3 & 1 & 1\\
%1 & 2 & 1\\
%0 & 0 & 3
%\end{bmatrix} & \begin{aligned}
%|3| &  >|1|+|0|\\
%|2| &  >|1|+|0|\\
%|3| &  >|1|+|1|
%\end{aligned} & \text{è a p.d.s.c}
%\end{array}
%\end{equation*}


Esistono delle particolari classi di matrici in cui la fattorizzazione $\displaystyle LU$ esiste ed è unica:
\begin{theorem}
[Condizioni sufficienti per fattorizzazione $\displaystyle LU$]
\index{fattorizzazione!LU!condizioni sufficienti}
Sia $\displaystyle A\in \mathbb{R}^{n\times n}$ non singolare:
\begin{itemize}
\item Se $\displaystyle A$ è a predominanza diagonale stretta per righe o per colonne allora $\displaystyle \exists !$ fattorizzazione $\displaystyle LU$.
\item Se $\displaystyle A$ è SDP (simmetrica e definita positiva) allora $\displaystyle \exists !$ fattorizzazione $\displaystyle LU$.
\end{itemize}
\label{thm:CS-LU}
\end{theorem}

% \textit{[Lezione 4 (16-03-2020)]}
\section{Metodo di eliminazione gaussiana (MEG)}
Presentiamo ora il metodo ideato da Gauss per la risoluzione di sistemi lineari.
L'idea generale è manipolare in maniera incrementale la matrice di partenza in modo da renderla, dopo $\displaystyle n$ passaggi, una matrice triangolare superiore, da risolvere successivamente con il metodo di sostituzione all'indietro. In un sistema lineare possiamo sommare e sottrarre tra loro due righe e moltiplicare due righe per uno scalare non nullo, lasciando inalterato il sistema. Vogliamo rendere nulli tutti gli elementi sulla prima colonna, dalla seconda riga in poi; poi sulla seconda colonna, dalla terza riga in poi e così via.
\begin{gather*}
A=A^{(1)}\rightarrow A^{(2)}\rightarrow A^{(3)}\rightarrow \cdots \rightarrow A^{(k)}\rightarrow \cdots \rightarrow A^{(n)}\\
\\
\underbrace{\begin{bmatrix}
a^{(1)}_{11} & \dotsc  & \dotsc  & \dotsc  & \dotsc  & a^{(1)}_{1n}\\
\vdots  &  &  &  &  & \vdots \\
\vdots  &  &  &  &  & \vdots \\
\vdots  &  &  &  &  & \vdots \\
\vdots  &  &  &  &  & \vdots \\
a^{(1)}_{n1} & \dotsc  & \dotsc  & \dotsc  & \dotsc  & a^{(1)}_{nn}
\end{bmatrix}}_{A=A^{(1)}}\rightarrow \underbrace{\begin{bmatrix}
a^{(1)}_{11} & \dotsc  & \dotsc  & \dotsc  & \dotsc  & a^{(1)}_{1n}\\
0 & a^{(2)}_{22} & \dotsc  & \dotsc  & \dotsc  & a^{(2)}_{2n}\\
\vdots  & \vdots  &  &  &  & \vdots \\
\vdots  & \vdots  &  &  &  & \vdots \\
\vdots  & \vdots  &  &  &  & \vdots \\
0 & a^{(2)}_{n2} & \dotsc  & \dotsc  & \dotsc  & a^{(2)}_{nn}
\end{bmatrix}}_{A^{(2)}}\rightarrow \\
\rightarrow \dotsc \rightarrow A^{(k)} =\begin{bmatrix}
a^{(1)}_{11} & \dotsc  & \dotsc  & \dotsc  & \dotsc  & a^{(1)}_{1n}\\
0 & a^{(2)}_{22} &  &  &  & \vdots \\
\vdots  & \ddots  & \ddots  &  &  & \vdots \\
\vdots  &  & 0 & a^{(k)}_{kk} & \dotsc  & a^{(k)}_{kn}\\
\vdots  &  & \vdots  & \vdots  &  & \vdots \\
0 & \dotsc  & 0 & a^{(k)}_{nk} & \dotsc  & a^{(k)}_{nn}
\end{bmatrix}.
\end{gather*}
Al passo $\displaystyle n$-esimo (l'ultimo) la matrice avrà la forma richiesta al fattore $U$:
\begin{equation}
  A^{(n)} =
  \begin{bmatrix}
      a^{(1)}_{11} & a^{(1)}_{12} & a^{(1)}_{13} & \dotsc & a^{(1)}_{1n} \\
      0             & a^{(2)}_{22} & a^{(2)}_{23} & \dotsc & a^{(2)}_{2n} \\
      \vdots        & \ddots        & a^{(3)}_{33} & \dotsc & a^{(3)}_{3n} \\
      \vdots        &               & \ddots        & \ddots & \vdots        \\
      0             & \dotsc        & \dotsc        & 0      & a^{(n)}_{nn}
  \end{bmatrix}.
  \label{eq:costruzione-U-con-MEG}
\end{equation}
Per fare ciò, ricaveremo una serie di \textit{moltiplicatori} $m_{ik}$ dividendo ogni elemento della riga per l'elemento pivotale, cioè quello sulla diagonale.
In particolare, vediamo lo pseudo-codice dell'algoritmo, che processerà la $k$-esima riga al $k$-esimo passo:

\begin{algo}
    \For{$k=1$ \KwTo $n-1$}{
    	\For{$i=k+1$ \KwTo $n$}{
    		\eIf{$a_{kk}=0$}{
    			termina algoritmo senza successo\;
    		}{
    			$m_{ik}=\frac{a_{ik} }{a_{kk} } $\;
    			\For{$j=k$ \KwTo $n$}{
    				$a_{ij} = a_{ij}-m_{ik}\cdot a_{kj} $\;
    			}
    		}
    	}
    }
    \caption{Metodo di Eliminazione Gaussiana}
\end{algo}
Ovviamente, l'algoritmo prosegue solo fintanto che non troviamo un \textbf{elemento pivotale}\index{elemento pivotale} nullo, ovvero $a_{kk} = 0$, visto che è il numero per cui dividiamo per trovare i moltiplicatori $m_{ik}$.


\textit{Esempio.}
Indichiamo con $(I), (II)$, etc la prima, seconda, etc riga della matrice.
\begin{equation*}
\underbrace{\begin{bmatrix}
1 & 2 & 3\\
2 & 2 & 2\\
3 & 7 & 4
\end{bmatrix}}_{A^{(1)}}\xrightarrow[(III)=(III)-3I]{(II)=(II)-2I}\underbrace{\begin{bmatrix}
1 & 2 & 3\\
0 & -2 & -4\\
0 & 1 & -5
\end{bmatrix}}_{A^{(2)}}\xrightarrow{(III)=(III)+\frac{(II)}{2}}\underbrace{\begin{bmatrix}
1 & 2 & 3\\
0 & -2 & -4\\
0 & 0 & -7
\end{bmatrix}}_{A^{(3)}}
\end{equation*}
\textbf{Osservazione. }

Al passo $k$-esimo del primo ciclo sono necessarie:
\begin{itemize}
\item $( n-k)$ divisioni per il calcolo dei moltiplicatori,
\item $( n-k)( n-k+1)$ moltiplicazioni e $( n-k)( n-k+1)$ somme per il calcolo di $a_{ij}$,
\item $( n-k)$ moltiplicazioni per il calcolo di $b_{j}$.
\end{itemize}
Si ha quindi un costo computazionale totale di:
\begin{equation*}
\sum\limits ^{n-1}_{k=1}[( n-k) +2( n-k)( n-k+1) +( n-k)] \approx \frac{2}{3} n^{3} \ \text{operazioni},
\end{equation*}
questo usando le identità
\begin{equation*}
\sum\limits ^{n}_{i=1} i=\frac{n( n+1)}{2} \quad \sum\limits ^{n}_{i=1} i^{2} =\frac{n( n+1)( 2n+1)}{6}.
\end{equation*}
A questo costo va aggiunto il costo della sostituzione all'indietro del sistema triangolare:
\begin{equation*}
\text{costo totale} = \text{eliminazione} +\text{sostituzione all'indietro} = \frac{2}{3} n^{3} +\frac{n^{2}}{2} \approx \frac{2}{3} n^{3}.
\end{equation*}

Osserviamo ora che il MEG può essere usato non solo per determinare la soluzione sistema lineare, ma anche per determinare la fattorizzazione $LU$. Il passo di eliminazione porta come risultato una matrice triangolare superiore, che possiamo pensare come la candidata a diventare la matrice $U$.
Per ricavare $\displaystyle L$, non resta che trovare l'unica matrice che moltiplicata per $\displaystyle U$ produce $\displaystyle A$.
Sotto le ipotesi di uno dei teoremi di unicità sopra citati, siamo in grado affermare che $L$ è effettivamente unica.
C'è modo di modificare l'algoritmo in modo da tenere traccia anche della matrice $\displaystyle L$ per non doverla ricostruire a posteriori. Si può dimostrare che infatti $L$ è la matrice dei moltiplicatori:
\begin{equation*}
L=\begin{bmatrix}
1 & 0 & \dotsc  & 0\\
m_{21} & 1 & \ddots  & \vdots \\
\vdots  & \ddots  & \ddots  & 0\\
m_{n1} & \dotsc  & m_{n,n-1} & 1
\end{bmatrix}.
\end{equation*}

In termini di costo computazionale, le operazioni svolte finora si riassumono come segue:
\begin{center}
\begin{tabular}{cc}
\toprule
 costo & azione da eseguire \\
\midrule
 $\approx 2n^{2}$ & risoluzione di $2$ sistemi triangolari \\
$\approx 2/3 \ n^{3}$ & trovare la fattorizzazione con MEG \\
\hline
$\approx 2/3 \ n^{3}$ & risoluzione complessiva del sistema \\
 \bottomrule
\end{tabular}

\end{center}
La soluzione presentata è quindi molto più vantaggiosa del costo fattoriale della regola di Cramer.

\section{Tecniche di Pivoting}

\textit{Esempio.}
Consideriamo ora una matrice
\begin{equation*}
A=\begin{bmatrix}
1 & 1 & 3\\
2 & 2 & 2\\
3 & 6 & 4
\end{bmatrix},
\end{equation*}
facciamo un passo del MEG:
\begin{equation*}
A^{(2)} =\begin{bmatrix}
1 & 1 & 3\\
0 & 0 & -1\\
0 & 3 & -5
\end{bmatrix}.
\end{equation*}
L'elemento pivotale $\displaystyle a^{(2)}_{22} =0$ quindi il MEG si interrompe.
Per risolvere questo intoppo, introduciamo le tecniche di \textbf{pivoting}\index{pivoting}.

L'idea in questo caso è scambiare la II e III riga.
Si noti inoltre che il minore di nord-ovest di ordine 2 della matrice $A$ è nullo:
\begin{equation*}
\operatorname{det}\begin{bmatrix}
1 & 1 \\
2 & 2
\end{bmatrix} = 0.
\end{equation*}
È quindi violata la condizione necessaria e sufficiente per la fattorizzazione $LU$. Dobbiamo trovare un'altra disposizione di righe o colonne, creando una nuova matrice $PA$ che verifichi la condizione e ci permetta di fare MEG per trovare la fattorizzazione $LU$, pre-moltiplicando $A$ per una matrice $P$ appropriata.
In generale, nell'implementazione di un algoritmo dovremo includere precise istruzioni per il calcolatore su come effettuare il pivoting.

Ipotizziamo che al passo $\displaystyle k$ del MEG troviamo l'elemento pivotale $\displaystyle a^{(k)}_{kk} =0$:
\begin{equation*}
\begin{bmatrix}
a^{(1)}_{11} & \dotsc  & \dotsc  & \dotsc  & a^{(1)}_{1n}\\
 & \ddots  &  &  & \vdots \\
 &  & a^{(k)}_{kk} =0 & \dotsc  & a^{(k)}_{kn}\\
 &  & \begin{array}{c}
\vdots \\
\vdots \\
\end{array} &  &
\end{bmatrix}.
\end{equation*}
Scorriamo la $k$-esima colonna finché troviamo un elemento non nullo. Ancora meglio, troviamo l'elemento \textit{più grande in modulo} della colonna, e quello corrisponderà alla riga che vogliamo scambiare con la $k$-esima.
Questa scelta dell'elemento è giustificata dal fatto che il calcolatore fa degli errori di approssimazione dato che deve usare un sistema floating-point. Si può vedere che l'operazione di divisione è ad alto rischio se il denominatore è molto piccolo, per cui dato che l'elemento pivotale, come visto, andrà a denominatore, scegliamo l'elemento in modulo maggiore, così da minimizzare gli errori di rappresentazione.

Questa tecnica prende il nome di MEG con \textbf{pivoting per righe}\index{pivoting!per righe}.
Scambiare righe di $A$ è equivalente a pre-moltiplicare per una appropriata matrice $P$.
Troviamo quindi una nuova matrice $PA$ tale che
\begin{equation*}
PA=LU.
\end{equation*}
$\displaystyle P$ è detta \textbf{matrice di permutazione}\index{matrice!di permutazione} ed è una matrice binaria, ossia composta da soli 0 e 1, che tiene traccia degli scambi di riga.
Nell'esempio di prima, per tenere traccia dello scambio tra la II e III riga, si ha la seguente matrice $P$:
\begin{equation*}
P=\begin{bmatrix}
1 & 0 & 0\\
0 & 0 & 1\\
0 & 1 & 0
\end{bmatrix}.
\end{equation*}
Il problema iniziale diventa allora
\begin{equation*}
PA\x =P\mathbf{b} \ \ \Leftrightarrow \ \ L\underbrace{U\x}_{\y} =P\mathbf{b} \ \ \Leftrightarrow \ \ \begin{cases}
L\y =P\mathbf{b}\\
U\x =\y.
\end{cases}
\end{equation*}
Similmente, esiste anche il \textbf{pivoting per colonne}\index{pivoting!per colonne}, ma MATLAB raramente lo utilizza.
\begin{equation*}
A^{(k)} =\begin{bmatrix}
a^{(1)}_{11} & \dotsc  & \dotsc  & \\
 & \ddots  &  & \\
 &  & a^{(k)}_{kk} =0 & \begin{array}{c c}
\cdots  & \cdots \\
\end{array}\\
 &  & \vdots  & \\
 &  & a^{(k)}_{nk} &
\end{bmatrix}
\end{equation*}
Un punto a cui bisogna dedicare particolare attenzione è che il pivoting per colonne \textit{cambia l'ordine delle incognite}. Il pivoting per colonne induce una scomposizione del tipo
\begin{equation*}
AQ=LU
\end{equation*}
con $\displaystyle Q$ matrice di permutazione. Il problema iniziale diventa allora
\begin{equation*}
A\x =\mathbf{b} \ \ \Leftrightarrow \ \ \underbrace{AQ}_{LU}\underbrace{Q^{T}\x}_{\mathbf{z}} =\mathbf{b} \ \ \Leftrightarrow \ \ L\underbrace{U\mathbf{z}}_{\y} =\mathbf{b} \ \ \Leftrightarrow \ \ \begin{cases}
L\y =\mathbf{b}\\
U\mathbf{z} =\y\\
Q^{T}\x =\mathbf{z}.
\end{cases}
\end{equation*}
Infine, è possibile anche applicare contemporaneamente il pivoting per righe e per colonne.

\section{Casi particolari di Fattorizzazioni LU}
In questa sezione presentiamo alcuni casi particolari di matrici in cui è particolarmente comodo trovare, e gestire a livello di memoria, la fattorizzazione $LU$.
Vedremo prima le matrici SDP e poi le matrici tridiagonali.

\subsection{Matrici simmetriche e definite positive}

Supponiamo che $\displaystyle A$ sia simmetrica e definita positiva.
In tal caso sappiamo dal teorema già visto \ref{thm:CS-LU} che esiste ed è unica la fattorizzazione $LU$.
In questo caso, essa prende il nome di fattorizzazione di Cholesky.
\begin{theorem}
[Fattorizzazione di Cholesky]
\index{fattorizzazione!di Cholesky}
Sia $\displaystyle A$ SDP. Allora $\displaystyle \exists !$ matrice $\displaystyle H$ triangolare inferiore con elementi positivi sulla diagonale tale che
\begin{equation*}
A=HH^{T}.
\end{equation*}
\end{theorem}
Questo equivale a pensare la risoluzione del sistema come:
\begin{equation*}
A\x =\mathbf{b} \ \ \Leftrightarrow \ \ \begin{cases}
H\y =\mathbf{b}\\
H^{T}\x =\y.
\end{cases}
\end{equation*}
\textit{Osservazione.} Possiamo sfruttare lo spazio di memoria che avevamo già allocato per tenere sia la matrice $\displaystyle A$ sia il suo fattore di Cholesky $\displaystyle H$:

\begin{equation*}
\begin{bmatrix}
\ddots  &  & A\\
 & \ddots  & \\
H &  & \ddots
\end{bmatrix},
\end{equation*}
in particolare, è \textit{sempre} conveniente quando la matrice è SDP.

\begin{algo}
    $h_{11} =\sqrt{a_{11}}$\;
    \For{$i=2$ \KwTo $n$}{
        \For{$j=1$ \KwTo $i-1$}{
            $h_{ij} =\frac{1}{h_{jj}}\left( a_{ij} -\sum\limits ^{j-1}_{k=1} h_{ik} h_{jk}\right)$\;
        }
        $h_{ii} =\sqrt{a_{ii} -\sum\limits ^{i-1}_{k=1} h^{2}_{ik}}$\;
    }
    \caption{Fattorizzazione di Cholesky}
\end{algo}
Questo algoritmo costa $\displaystyle \approx \frac{n^{3}}{3}$, circa la metà del MEG.

\subsection{Matrici tridiagonali}
% \textit{[Lezione 5 (17-03-2020)]}

Supponiamo che $\displaystyle A$ sia tridiagonale, ovvero che abbia questa forma:
\begin{equation*}
    A=
    \begin{bmatrix}
        a_{1} & c_{1} &        &        & 0       \\
        b_{2} & a_{2} & c_{2}  &        &         \\
              & b_{3} & a_{3}  & \ddots &         \\
              &       & \ddots & \ddots & c_{n-1} \\
        0     &       &        & b_{n}  & a_{n}
    \end{bmatrix}.
\end{equation*}
Nel caso di matrici tridiagonali si usa l'\textbf{algoritmo di Thomas}\index{algoritmo!di Thomas}\index{fattorizzazione!di Thomas} che scrive la fattorizzazione e istruisce su come risolvere i due sistemi a valle della fattorizzazione.
Necessitiamo solo di $3$ vettori per memorizzare completamente la matrice:
\begin{align*}
\mathbf{a} & =[ a_{1} ,\dotsc ,a_{n}]^{T} \in \mathbb{R}^{n}\\
\mathbf{b} & =[ b_{2} ,\dotsc ,b_{n}]^{T} \in \mathbb{R}^{n-1}\\
\mathbf{c} & =[ c_{1} ,\dotsc ,c_{n-1}]^{T} \in \mathbb{R}^{n-1}.
\end{align*}
Si dimostra che i fattori $\displaystyle L$ ed $U$ di $\displaystyle A$ hanno la seguente struttura:
\begin{equation*}
L=\begin{bmatrix}
1 & 0 & \dotsc  & \dotsc  & 0\\
\beta _{2} & 1 &  &  & \vdots \\
0 & \beta _{3} & 1 &  & \vdots \\
\vdots  & \ddots  & \ddots  & \ddots  & 0\\
0 & \dotsc  & 0 & \beta _{n} & 1
\end{bmatrix} \qquad U=\begin{bmatrix}
\alpha _{1} & c_{1} & 0 & \dotsc  & 0\\
0 & \alpha _{2} & c_{2} & \ddots  & \vdots \\
\vdots  &  & \alpha _{3} & \ddots  & 0\\
\vdots  &  &  & \ddots  & c_{n-1}\\
0 & \dotsc  & \dotsc  & 0 & \alpha _{n}
\end{bmatrix}.
\end{equation*}
Si noti che i termini $c_{k}$ sono rimasti invariati.
Per determinare i fattori $\displaystyle L$ ed $U$ è quindi sufficiente determinare:
\begin{equation*}
\alpha _{1} ,\alpha _{2} ,\dotsc ,\alpha _{n} \quad \text{e} \quad \beta _{1} ,\beta _{2} ,\dotsc ,\beta _{n}.
\end{equation*}

Questi si possono determinare risolvendo col seguente algoritmo. \\

\begin{algo}
    $\alpha _{1} =a_{1}$\;
    \For{$i=2$ \KwTo $n$}{
        $\beta _{i} =\frac{b_{i}}{\alpha _{i-1}}$\;
        $\alpha _{i} =a_{i} -\beta _{i} c_{i-1}$\;
    }
    \caption{Algoritmo di Thomas}
\end{algo}
Possiamo usare queste relazioni per risolvere direttamente i sistemi triangolari a valle della fattorizzazione trovata:
\begin{itemize}
\item per risolvere $L\y =\mathbf{b}$
\begin{equation*}
y_{1} =b_{1} ,\quad y_{i} =b_{i} -\beta _{i} y_{i-1} ,\quad i=2,\dotsc ,n,
\end{equation*}
\item per risolvere $U\x =\y$
\begin{equation*}
x_{n} =\frac{y_{n}}{\alpha _{n}} ,\quad x_{i} =\frac{y_{i} -c_{i} x_{i+1}}{\alpha _{i}} ,\quad i=n-1,\dotsc ,1.
\end{equation*}
\end{itemize}

Usare matrici tridiagonali è particolarmente vantaggioso perché abbatte il costo computazionale a un numero di operazioni proporzionale a $n$, invece che a $n^2$ o più.
In particolare, si hanno $8n-7$ operazioni.

\textit{Esempio }$( n=3)$\textit{. }Assegnati $a_{1} ,a_{2} ,a_{3} ,c_{1} ,c_{2} ,l_{2} ,l_{3}$ trovare $\alpha _{1} ,\alpha _{2} ,\alpha _{3} ,\beta _{2} ,\beta _{3}$.
\begin{equation*}
\underbrace{\begin{bmatrix}
a_{1} & c_{1} & 0\\
{l_{2}} & a_{2} & c_{2}\\
0 & l_{3} & a_{3}
\end{bmatrix}}_{A} =\underbrace{\begin{bmatrix}
1 & 0 & 0\\
\beta _{2} & 1 & 0\\
0 & \beta _{3} & 1
\end{bmatrix}}_{L}\underbrace{\begin{bmatrix}
\alpha _{1} & c_{1} & 0\\
0 & \alpha _{2} & c_{2}\\
0 & 0 & \alpha _{3}
\end{bmatrix}}_{U}
\end{equation*}
Allora
\begin{equation*}
\begin{array}{ r c l }
a_{1} =1\cdot \alpha _{1} & \Rightarrow  & \alpha _{1} =a_{1}\\
l_{2} =\beta _{2} \alpha _{1} & \Rightarrow  & \beta _{2} =l_{2} /\alpha _{1}\\
a_{2} =\beta _{2} c_{1} +\alpha _{2} & \Rightarrow  & \alpha _{2} =a_{2} -\beta _{2} c_{1}\\
l_{3} =\beta _{3} \alpha _{2} & \Rightarrow  & \beta _{3} =l_{3} /\alpha _{2}\\
a_{3} =\beta _{3} c_{2} +\alpha _{3} & \Rightarrow  & \alpha _{3} =a_{3} -\beta _{3} c_{2}.
\end{array}
\end{equation*}
\section{Condizionamento di una matrice}

Gli algoritmi che abbiamo visto finora manipolano la matrice interessata.
Questo comporta che il calcolatore debba fare delle operazioni floating-point, a cominciare dal memorizzare certi numeri.
Queste operazioni saranno inevitabilmente approssimazioni, in quanto è spesso impossibile avere calcoli esattamente corretti.
Per esempio, un numero irrazionale dovrà essere troncato visto che ha infinite cifre decimali, e in particolare dopo circa $16$ cifre dopo la virgola in una normale architettura floating-point.

È importante, quindi, capire quanto queste approssimazioni influiscano sul risultato finale. Questi metodi, tra cui la fattorizzazione $LU$, cadono nella categoria dei \textit{metodi diretti}. Assumendo di non manipolare la matrice, possiamo invece usare dei \textbf{metodi indiretti}\index{metodi indiretti} che ci permettono di approssimare la \textit{soluzione esatta} $\x$ con una \textbf{soluzione approssimata}\index{soluzione!approssimata} $\tilde{\x}$.
In tal caso è però anche necessario saper stimare dall'alto l'errore: $\left\Vert \x -\tilde{\x}\right\Vert$ deve essere minore di una certa quantità nota, ovvero che non dipenda dalla soluzione esatta che è incognita.
In generale, l'errore deve anche poter essere ridotto a piacere, sotto una soglia di tolleranza arbitrariamente fissata.

La soluzione approssimata $\tilde{x}$ può essere vista come la soluzione esatta di un certo \textbf{sistema perturbato}\index{sistema!perturbato}.

Il \textbf{numero di condizionamento}\index{numero di condizionamento} di una matrice rivela quanto essa sia ``buona'' o ``cattiva'', ossia in quale misura eseguire operazioni su di essa provochi errori di rappresentazione e computazione.

Il \textbf{teorema di stabilità}\index{teorema!di stabilità}\index{stabilità} riassumerà quando una matrice è ``buona'' o ``cattiva'' e come ciò influenzi il risultato del calcolatore rispetto alla soluzione esatta.
\begin{definition}
[Norma $L^p$ di un vettore]
\index{norma!$L^p$!di un vettore}
Sia $\displaystyle \mathbf{z} \in \mathbb{R}^{n}$. Definiamo la sua norma $L^p$ come:
\begin{equation*}
\Vert \mathbf{z}\Vert _{p} =\left(\sum ^{n}_{i=1} |z_{i} |^{p}\right)^{1/p} \ \ \forall p\in [ 1,+\infty ).
\end{equation*}
\label{def:norma-vettore}
\end{definition}
\textit{Esempi di norme di vettore.}
\begin{enumerate}
\item Se $p=\infty $, si ha $\Vert \mathbf{z}\Vert _{\infty } =\max_{i=1,\dotsc ,n} |z_{i} |$.
\item Se $\displaystyle p=2$, si ha la \textbf{norma euclidea} (vedi \ref{def:norma-euclidea}).
\end{enumerate}

Possiamo usare la definizione di norma di vettore ed estenderla a norma di matrice.
\begin{definition}
[Norma $L^p$ di una matrice]
\index{norma!$L^p$!di una matrice}
Sia $\displaystyle A\in \mathbb{R}^{n\times n}$. Definiamo la sua norma $L^p$ come:
\begin{equation*}
\Vert A\Vert _{p} =\sup _{\substack{\mathbf{z} \in \mathbb{R}^{n} \\ \mathbf{z} \neq \mathbf{0}} } \frac{\Vert A\mathbf{z}\Vert _{p}}{\Vert \mathbf{z}\Vert _{p}} \quad \forall p\in [ 1,+\infty ).
\end{equation*}
È detta \textbf{norma indotta}, cioè è indotta da quella vettoriale.
\end{definition}
\textit{Esempi di norme matriciali.}
\begin{itemize}
    \item Se $p=1$, si ha:
    \begin{equation*}
        \Vert A\Vert _{1} =\max_{j=1,\dotsc,n}\sum\limits ^{n}_{i=1} |a_{ij}|,
    \end{equation*}
    cioè il massimo della somma delle colonne in modulo.
    \item Se $p=\infty$, si ha:
    \begin{equation*}
      \Vert A\Vert _{\infty } =\max_{i=1,\dotsc,n}\sum\limits ^{n}_{j=1} |a_{ij}|,
    \end{equation*}
    cioè il massimo della somma delle righe in modulo.
    \item Se $p=2$, si ha $\Vert A\Vert _{2} =\sqrt{\rho \left( A^{T} A\right)} \equiv \sqrt{\rho \left( A A^{T}\right)}$, definendo $\rho$ come a seguire.
\end{itemize}
\begin{definition}
Siano $\lambda _{i}$ gli autovalori di $M$. Definiamo il \textbf{raggio spettrale}\index{raggio spettrale} di $\displaystyle M$
\begin{equation*}
\rho (M) =\max_{i=1,\dotsc ,n} |\lambda _{i} |.
\end{equation*}
\end{definition}
Se $\displaystyle p=2$, $\Vert A\Vert _{2}$ si chiama \textbf{norma spettrale} e ha le seguenti proprietà:
\begin{itemize}
\item se $A$ è SDP, allora $\Vert A\Vert _{2} =\sqrt{\rho \left( A^{2}\right)} =\sqrt{\rho (A)^{2}} =\rho (A) =\lambda _{\text{max}}(A)$,
\item $\Vert A\Vert _{2} \geqslant 0$,
\item $\Vert \alpha A\Vert _{2} =\alpha \Vert A\Vert _{2}$,
\item $\Vert A+B\Vert _{2} \leqslant \Vert A\Vert _{2} +\Vert B\Vert _{2}$,
\item $\Vert A\mathbf{v}\Vert _{2} \leqslant \Vert A\Vert _{2}\Vert \mathbf{v}\Vert _{2}$ con $\mathbf{v}$ vettore.
\end{itemize}

\subsection{Il numero di condizionamento}
\spazioSecBox
\begin{definition}
[Numero di condizionamento]
\index{numero!di condizionamento}
Sia $\displaystyle A\in \mathbb{R}^{n\times n}$ invertibile. Definiamo il numero di condizionamento in norma $\displaystyle p$ come:
\begin{equation*}
K_{p}(A) =\Vert A\Vert _{p} \ \left\Vert A^{-1}\right\Vert _{p} \quad p=1,2,\dotsc ,+\infty.
\end{equation*}
Se $\displaystyle A$ non è invertibile, lo si definisce come $K_{p}(A) =  \infty$.
\end{definition}
Il numero di condizionamento riassume quanto una matrice è ``buona'', o ``ben condizionata'', nel senso spiegato prima.
Tanto più il numero condizionamento della matrice è grande, tanto più è ``cattiva'', perché si avvicina ad una matrice singolare.

Possiamo misurare $K_p$ rispetto a un qualsiasi $p$, ma solitamente useremo $p=2$.
In tal caso esso prende il nome di \textbf{numero di condizionamento spettrale}\index{numero di condizionamento!spettrale}.

\textbf{Proprietà.}
\begin{enumerate}
\item $\displaystyle K_{p}(A) \geqslant K_{p}(I) =1$: più è basso e più la matrice si comporta come l'identità\footnote{``La matrice più \textit{bella} di tutte''.}.
\item $\displaystyle K_{p}(A) =K_{p}\left( A^{-1}\right)$: una matrice è ``brutta'' tanto quanto la sua inversa.
\item $\displaystyle K_{p}( \alpha A) =K_{p}(A) ,\ \ \forall \alpha \in \mathbb{R} ,\alpha \neq 0$: se abbiamo una matrice ``buona'' e la moltiplichiamo per uno scalare, essa continua a rimanere ``buona''.
\end{enumerate}

\needspace{3\baselineskip}
\textbf{Osservazione (condizionamento spettrale).}

Se $A$ è SDP, allora:
\begin{equation*}
K_{2}(A) =\underbrace{\Vert A\Vert _{2}}_{\lambda _{\text{max}}(A)} \ \ \underbrace{\left\Vert A^{-1}\right\Vert _{2}}_{\frac{1}{\lambda _{\text{min}}(A)}} =\frac{\lambda _{\text{max}}(A)}{\lambda _{\text{min}}(A)}.
\end{equation*}
Più gli autovalori estremali, cioè il minimo e il massimo, di $\displaystyle A$ sono lontani tra di loro, più $\displaystyle A$ è una matrice mal condizionata. Il caso ottimale è che tutti gli autovalori siano uguali, in cui il loro rapporto è unitario.
\begin{definition}[Distanza $L^p$]
Sia $\displaystyle A\in \mathbb{R}^{n\times n}$ non singolare. Definiamo la distanza $L^p$ di $A$ come:
\begin{equation*}
\text{dist}_{p}(A) =\min\left\{\frac{\Vert \delta A\Vert _{p}}{\Vert A\Vert _{p}} :\ A+\delta A\ \text{è singolare}\right\}.
\end{equation*}
\end{definition}
Si può dimostrare che la distanza è inversamente proporzionale al numero di condizionamento:
\begin{equation*}
\operatorname{dist}_{p}(A) =\frac{1}{K_{p}(A)}.
\end{equation*}
\section{Analisi di stabilità}

Siano:
\begin{itemize}
\item $\displaystyle A\x =\mathbf{b}$ il \textbf{problema originale (PO).}
\item $\displaystyle ( A+\delta A)(\x +\delta \x) =\mathbf{b} +\delta \mathbf{b}$ il \textbf{problema perturbato (PP)}, quello che risolviamo con un calcolatore, introducendo perturbazioni per ciascuno dei tre elementi che lo compongono.
\end{itemize}

\begin{theorem}
[stabilità]
\index{teorema!di stabilità}
\index{stabilità}
Sia $\displaystyle A\in \mathbb{R}^{n\times n}$ non singolare. Sia inoltre $\displaystyle \delta A\in \mathbb{R}^{n\times n}$ una perturbazione tale che $\displaystyle \left\Vert A^{-1}\right\Vert _{p}\Vert \delta A\Vert _{p} < 1$ per una generica norma indotta $\displaystyle p$.
Allora se $\displaystyle \x \in \mathbb{R}^{n}$ è soluzione di (PO) con $\displaystyle \mathbf{b} \in \mathbb{R}^{n} ,\ \mathbf{b} \neq \mathbf{0}$ e $\displaystyle \x +\delta \x \in \mathbb{R}^{n}$ è soluzione di (PP) per $\displaystyle \delta \mathbf{b} \in \mathbb{R}^{n}$, si ha
\begin{equation*}
\frac{\Vert \delta \x\Vert _{p}}{\Vert \x\Vert _{p}} \leqslant \left[\frac{K_{p}(A)}{1-K_{p}(A)\frac{\Vert \delta A\Vert _{p}}{\Vert A\Vert _{p}}}\right]\left(\frac{\Vert \delta \mathbf{b}\Vert _{p}}{\Vert \mathbf{b}\Vert _{p}} +\frac{\Vert \delta A\Vert _{p}}{\Vert A\Vert _{p}}\right).
\end{equation*}
\label{thm:teorema-di-stabilita}
\end{theorem}
\textit{Dimostrazione.}

Si ha innanzitutto che:
\begin{equation*}
\left\Vert A^{-1} \delta A\right\Vert _{p} \leqslant \underbrace{\left\Vert A^{-1}\right\Vert _{p} \ \Vert \delta A\Vert _{p} < 1}_{\text{per ipotesi}}.
\end{equation*}
Si può inoltre dimostrare che $\displaystyle I+A^{-1} \delta A$ è invertibile se vale l'ipotesi considerata. Inoltre, vale anche la seguente disuguaglianza:
\begin{equation}
\left\Vert \left( I+A^{-1} \delta A\right)^{-1}\right\Vert _{p} \leqslant \frac{1}{1-\left\Vert A^{-1}\right\Vert _{p} \ \Vert \delta A\Vert _{p}}.
\label{eq:dis-in-teo-stab}
\end{equation}
A questo punto prendiamo il problema perturbato (PP):
\begin{align*}
( A+\delta A)(\x +\delta \x) & =\mathbf{b} +\delta \mathbf{b} & \\
A\x +\delta A\x +A\delta \x +\delta A\delta \x & =\mathbf{b} +\delta \mathbf{b} & \\
\delta A\x +A\delta \x +\delta A\delta \x & =\delta \mathbf{b} & \text{(ricordando che $A\x =\mathbf{b}$)}\\
( A+\delta A) \delta \x & =\delta \mathbf{b} -\delta A\x & \\
A^{-1}[( A+\delta A) \delta \x] & =A^{-1}[ \delta \mathbf{b} -\delta A\x] & \text{(moltiplicando per $A^{-1}$)}\\
\left( I+A^{-1} \delta A\right) \delta \x & =A^{-1} \delta \mathbf{b} -A^{-1} \delta A\x. &
\end{align*}
Ricordiamoci che possiamo, per ipotesi, moltiplicare per $\left( I+A^{-1} \delta A\right)^{-1}$, ottenendo:
\begin{equation*}
\delta \x =\left( I+A^{-1} \delta A\right)^{-1}\left( A^{-1} \delta \mathbf{b} -A^{-1} \delta A\x\right).
\end{equation*}
Passando alle norme:
\begin{equation*}
\Vert \delta \x\Vert _{p} \leqslant \left\Vert \left( I+A^{-1} \delta A\right)^{-1}\right\Vert _{p} \ \left\Vert A^{-1} \delta \mathbf{b} -A^{-1} \delta A\x\right\Vert _{p}.
\end{equation*}
Usando la disuguaglianza triangolare delle norme: %essendo la norma della differenza più piccola della somma delle norme
\begin{equation*}
\Vert \delta \x\Vert _{p} \leqslant \left\Vert \left( I+A^{-1} \delta A\right)^{-1}\right\Vert _{p}\left(\left\Vert A^{-1} \delta \mathbf{b}\right\Vert _{p} +\left\Vert A^{-1} \delta A\x\right\Vert _{p}\right).
\end{equation*}
Usiamo ora la disuguaglianza \eqref{eq:dis-in-teo-stab}:
\begin{equation*}
\Vert \delta \x\Vert _{p} \leqslant \frac{1}{1-\left\Vert A^{-1}\right\Vert _{p} \ \Vert \delta A\Vert _{p}}\left(\left\Vert A^{-1}\right\Vert _{p} \ \Vert \delta \mathbf{b}\Vert _{p} +\ \left\Vert A^{-1}\right\Vert _{p} \ \Vert \delta A\Vert _{p} \ \Vert \x\Vert _{p}\right).
\end{equation*}
Dividiamo per $\Vert \x\Vert _{p}$:
\begin{equation*}
\frac{\Vert \delta \x\Vert _{p}}{\Vert \x\Vert _{p}} \leqslant \frac{1}{1-\left\Vert A^{-1}\right\Vert _{p} \ \Vert \delta A\Vert _{p}}\left(\frac{\left\Vert A^{-1}\right\Vert _{p} \ \Vert \delta \mathbf{b}\Vert _{p}}{\Vert \x\Vert _{p}} +\ \left\Vert A^{-1}\right\Vert _{p} \ \Vert \delta A\Vert _{p}\right).
\end{equation*}
Raccogliendo $\left\Vert A^{-1}\right\Vert _{p}$ e dividendo e moltiplicando per $\Vert A \Vert_{p}$ si ottiene:
\begin{equation*}
\frac{\Vert \delta \x\Vert _{p}}{\Vert \x\Vert _{p}} \leqslant \frac{\overbrace{\left\Vert A^{-1}\right\Vert _{p} \ \Vert A \Vert_{p}}^{K_{p}(A)}}{1-\left\Vert A^{-1}\right\Vert _{p} \ \Vert \delta A\Vert _{p}}\left(\frac{\Vert \delta \mathbf{b}\Vert _{p}}{\Vert \x\Vert _{p}\Vert A \Vert_{p}} +\frac{\Vert \delta A\Vert _{p}}{\Vert A \Vert_{p}}\right).
\end{equation*}
Ricordiamo ora che:
\begin{equation*}
A\x =\mathbf{b} \ \ \Rightarrow \ \ \Vert \mathbf{b}\Vert _{p} =\Vert A\x\Vert _{p} \leqslant \Vert A\Vert _{p}\Vert \x\Vert _{p} \ \ \Rightarrow \ \ \frac{1}{\Vert A\Vert _{p}\Vert \x\Vert _{p}} \leqslant \frac{1}{\Vert \mathbf{b}\Vert _{p}}.
\end{equation*}
Abbiamo così stimato il primo termine dentro la parentesi. Ora moltiplichiamo e dividiamo a denominatore per $\Vert A \Vert_{p}$ ottenendo il numero di condizionamento $K_{p}(A)$ anche a denominatore:
\begin{equation*}
\frac{\Vert \delta \x\Vert _{p}}{\Vert \x\Vert _{p}} \leqslant \frac{K_{p}(A)}{1-\left\Vert A^{-1}\right\Vert _{p}\Vert \delta A\Vert _{p} \cdot \frac{\Vert A\Vert _{p}}{\Vert A\Vert _{p}}}\left(\frac{\Vert \delta \mathbf{b}\Vert _{p}}{\Vert \mathbf{b}\Vert _{p}} +\frac{\Vert \delta A\Vert _{p}}{\Vert A\Vert _{p}}\right)
\end{equation*}
ovvero:
\begin{gather*}
\frac{\Vert \delta \x\Vert _{p}}{\Vert \x\Vert _{p}} \leqslant \frac{K_{p}(A)}{1-K_{p}(A) \cdot \frac{\Vert \delta A\Vert _{p}}{\Vert A\Vert _{p}}}\left(\frac{\Vert \delta \mathbf{b}\Vert _{p}}{\Vert \mathbf{b}\Vert _{p}} +\frac{\Vert \delta A\Vert _{p}}{\Vert A\Vert _{p}}\right).\qed
\end{gather*}

\textit{Osservazione.} Se $\displaystyle \delta A=0$ si ha che:
\begin{equation*}
\frac{\Vert \delta \x\Vert _{p}}{\Vert \x\Vert _{p}} \leqslant K_{p}(A)\left(\frac{\Vert \delta \mathbf{b}\Vert _{p}}{\Vert \mathbf{b}\Vert _{p}}\right).
\end{equation*}
Ovvero, anche se la perturbazione è molto piccola ma il condizionamento della matrice è grande, ossia la matrice è molto ``brutta'', l'errore rispetto alla soluzione esatta rimane comunque non trascurabile.
Da ciò segue che:
\begin{corollario}
Se $\displaystyle \delta A=0$, si ha che:
\begin{equation*}
\frac{1}{K_{p}(A)}\frac{\Vert \delta \mathbf{b}\Vert _{p}}{\Vert \mathbf{b}\Vert _{p}} \leqslant \frac{\Vert \delta \x\Vert _{p}}{\Vert \x\Vert _{p}} \leqslant K_{p}(A) \ \frac{\Vert \delta \mathbf{b}\Vert _{p}}{\Vert \mathbf{b}\Vert _{p}}.
\end{equation*}
\end{corollario}
Cioè possiamo stimare l'errore anche dal basso.
Questo è notevole perché è spesso facile trovare stime dall'alto, mentre fornire stime dal basso è più difficile, o non sempre possibile.

\textit{Dimostrazione del corollario.} Scriviamo il problema perturbato con $\displaystyle \delta A=0$ e passiamo alle norme:
\begin{align*}
A\delta \x & =\delta \mathbf{b} & \text{((PP) con $\delta A=0$)}\\
\Vert \delta \mathbf{b}\Vert _{p} & =\Vert A\delta \x\Vert _{p} \leqslant \Vert A\Vert _{p}\Vert \delta \x\Vert _{p}. & \text{(passando alle norme)}
\end{align*}
Ricordiamo ora che:
\begin{equation*}
\Vert \x\Vert _{p} =\left\Vert A^{-1}\mathbf{b}\right\Vert _{p} \leqslant \left\Vert A^{-1}\right\Vert _{p} \ \Vert \mathbf{b}\Vert _{p}.
\end{equation*}
e moltiplichiamo quindi per $\Vert \x\Vert _{p}$:
\begin{align*}
\Vert \x\Vert _{p} \ \Vert \delta \mathbf{b}\Vert _{p} & \leqslant \Vert A\Vert _{p} \ \Vert \delta \x\Vert _{p} \ \underbrace{\left\Vert A^{-1}\right\Vert _{p} \ \Vert \mathbf{b}\Vert _{p}}_{\Vert \x\Vert _{p}}\\
\frac{\Vert \delta \mathbf{b}\Vert _{p}}{\Vert \mathbf{b}\Vert _{p}} & \leqslant \frac{\Vert \delta \x\Vert _{p}}{\Vert \x\Vert _{p}} \ \underbrace{\Vert A\Vert _{p} \ \left\Vert A^{-1}\right\Vert _{p} \ }_{K_{p}(A)}\\
\frac{1}{K_{p}(A)}\frac{\Vert \delta \mathbf{b}\Vert _{p}}{\Vert \mathbf{b}\Vert _{p}} & \leqslant \frac{\Vert \delta \x\Vert _{p}}{\Vert \x\Vert _{p}}.
\qed
\end{align*}

% \textit{[Lezione 6 (23-03-2020)]}
\section{Problema del fill-in}
Supponiamo di avere una matrice tridiagonale, che sappiamo poter fattorizzare con l'\textbf{algoritmo di Thomas}\index{algoritmo!di Thomas}.
La fattorizzazione $LU$ preserva questa struttura\textit{ a banda}:
\begin{equation*}
\underbrace{\begin{bmatrix}
\cdot  & \cdot  &  &  & \\
\cdot  & \cdot  & \cdot  &  & \\
 & \cdot  & \cdot  & \cdot  & \\
 &  & \cdot  & \cdot  & \cdot \\
 &  &  & \cdot  & \cdot
\end{bmatrix}}_{A}\xrightarrow{\text{Thomas}}\underbrace{\begin{bmatrix}
1 &  &  &  & \\
\cdot  & 1 &  &  & \\
 & \cdot  & 1 &  & \\
 &  & \cdot  & 1 & \\
 &  &  & \cdot  & 1
\end{bmatrix}}_{L}\underbrace{\begin{bmatrix}
\cdot  & \cdot  &  &  & \\
 & \cdot  & \cdot  &  & \\
 &  & \cdot  & \cdot  & \\
 &  &  & \cdot  & \cdot \\
 &  &  &  & \cdot
\end{bmatrix}}_{U}
\end{equation*}
abbiamo quindi a che fare con matrici quasi vuote.
\begin{definition}
[Matrice sparsa]
Diciamo che $A$ è \textbf{sparsa}\index{matrice!sparsa} se il numero di elementi non nulli (NNZ: Number of Non Zero) è circa $O(n)$.
\end{definition}
Se applichiamo la fattorizzazione $LU$ a una matrice sparsa generica (non a banda) i fattori $L$ e $U$ in generale \textit{non preservano la struttura di sparsità}: anzi, talvolta sono addirittura piene.
Per questa ragione, non possiamo fare previsioni su quanta memoria servirà per memorizzarle.
Questo è noto come \textbf{problema del fill-in}\index{problema!del fill-in}.
Per quantificare questo problema, si guarda il numero di elementi non zero di $A$ e di $LU$, e si fa la loro differenza.

\textit{Esempio.}
\begin{equation*}
	\underbrace{
		\begin{bmatrix}
				  &       & \cdot &       &   \\
				  &       &       &       &   \\
			\cdot &       &       & \cdot &   \\
				  & \cdot &       &       &   \\
				  &       & \cdot &       &
		\end{bmatrix}
		}_{A}\xrightarrow{\text{Fattorizzazione} \ LU}\underbrace{
		\begin{bmatrix}
			1     &       &       &   &   \\
			      & 1     &       &   &   \\
			      &       & 1     &   &   \\
			\cdot & \cdot &       & 1 &   \\
			\cdot &       & \cdot &   & 1
		\end{bmatrix}
		}_{L}\underbrace{
		\begin{bmatrix}
			\cdot &       &       & \cdot & \cdot \\
			      & \cdot &       & \cdot &       \\
			      &       & \cdot &       & \cdot \\
			      &       &       & \cdot &       \\
			      &       &       &       & \cdot
		\end{bmatrix}
	}_{U}
\end{equation*}
Illustriamo qualche possibile soluzione a questo problema:
\begin{enumerate}
\item Riordinare le righe e le colonne in modo da ottenere le righe e le colonne di $A$ in una configurazione che minimizza il fill-in (approccio utilizzato da MATLAB).
\item Cambiare completamente prospettiva, cercando di risolvere il problema \textit{senza manipolare affatto la matrice }$A$. L'idea sarà sviluppata nel prossimo capitolo nei cosiddetti \textbf{metodi iterativi}\index{metodi iterativi}.
\end{enumerate}
